% This is LLNCS.DOC the documentation file of
% the LaTeX2e class from Springer-Verlag
% for Lbecture Notes in Computer Science, version 2.4
\documentclass{llncs}
\usepackage{llncsdoc}
%
%\usepackage{aopmath}
\usepackage{graphicx}

\usepackage{xcolor}
\usepackage{amsmath}
\usepackage{url}
\usepackage{colortbl}
\usepackage{multirow}


\usepackage{amssymb}
%\newtheorem{definition}{Definition} % [section]
%\newtheorem{example}{Example} % [section]
\newcommand{\pivot}[1]{\mathbin{\, {#1} \,}}
\newcommand{\Pivot}[1]{\mathbin{\; {#1} \;}}
\newcommand{\Var}[0]{\mbox{\texttt{Var}}}
\let\from=\leftarrow

\newcommand{\keywords}[1]{\par\addvspace\baselineskip
\noindent\keywordname\enspace\ignorespaces#1}
\def\code#1{\texttt{#1}}

\begin{document}
%\bibliographystyle{acmtrans}

\long\def\comment#1{}

\title{Differentiable Concurrent Constraint Programs}

\author{{\sc Incomplete Draft}}
\institute{}
\maketitle

\begin{abstract}

\keywords{machine learning, concurrent constraint programming}
\end{abstract}
\section{Basic idea}

The {\em supervised machine learning} approach is applicable in
settings where the programmer is concerned with specifying a function
that must work accurately even in the presence of significant amounts
of noise in the input. For instance, suppose the programmer must write
code that labels an arbitrary input image with a tag (``dog with
hat'') that best characterizes the image. The space of variations in
the input is generally so vast and potentially so ill-understood
mathematically that it may simply not be possible for the user to
programmatically specify all the logic for the function.

Instead we desire an approach whereby (portions of the) logic may be
{\em learnt} by automatic techniques, given a (potentially very) large
collection of observations (input/output pairs) of the given
function. 

Concretely, we may describe the problem as follows. Instead of
providing a fully realized, executable function $f$, the programmer
specifies a space $\cal F$ of possible 
functions, obtained by varying the parameters $\theta$ of
some parametric function $f_{\theta}$.  Also
available is the set of {\em observations}
$(\bar{x}_i, \bar{y}_i)$ (for $i\in I$) specifying pairs of input values with their
associated output values, and a {\em loss function} $L$ which can take
two values $\bar{y},\bar{y}'$ and return a real value 
$L(\bar{y},\bar{y}')$ which measures how far one value is from the
other. The machine learning problem is now to determine a specific
value $\hat{\theta}$ for the parameters which minimizes
$$\Sigma_{i \in I} L(f_\theta(\bar{x}_i), \bar{y}_i)$$

Many algorithmic techniques are available, under various conditions,
to address this problem. For instance, in situations
where $f_{\theta}$ represents a differentiable function of its
parameters $\theta$, one may use stochastic gradient descent to find a
minima. Under other conditions, the Expectation Maximization algorithm
may be used.

A very powerful instance of this general picture is obtained with
(feed forward) {\em
  deep neural networks}. In this setting, the parametric function
$f_{\theta}$ is given by $k$ layers of compute elements (called
``neurons''), an element in layer $i$ connected to some (possibly all)
elements in the preceding layer. Each neuron is associated with a set
of weights $W$ (its parameters) used to linearly sum its inputs.
(There is also usually a bias value.) Some non-linear
(but differentiable) thresholding function (e.g. sigmoid, $\lambda
t.\,(1/(1+e^{-t}))$) is used to determine the output of this
element. One can think of the entire network as specified by a
$k$-nested functional term, involving summations and pointwise
applications of the thresholding function. It has been shown that any
arbitrary non-linear (real-valued) function can be approximated
arbitrarily closely with sufficiently many parameters, and sufficient
training input. 

What has made the machine learning problem of great interest in recent
years is the availability of large amounts of labeled training data,
and vast amounts of (CPU, GPU) computation that can be brought to bear
to train the given function. A better understanding of how training
algorithms such as asynchronous stochastic gradient networks can be
made to work for deep networks has brought about a startling increase
in accuracy in many applications, and a significant increase in the
range of applications amenable to these techniques.

\subsection{Learning concurrent constraint programs}
A significant drawback of deep neural networks is their {\em
  opacity}. In general, it is not possible to give a meaningful answer
to the question of {\em why} the learnt program does what it
does. At hand is simply the value of the (potentially billions of)
parameters, poor material from which to construct causally coherent
explanations accessible to a human observer.

Our intuition is that it will be easier to develop explanations in a
context in which the machine learning paradigm is applied to a
logic-based programming framework, leading to a natural combination of
symbolic computation and numeric approximation. 

In this paper we show how the machine learning problem can be
instantiated in the context of concurrent constraint programming.
(Our main interest is going to be to develop this theory for timed
CCP.) The major novelty of our approach is that, since it is based on
CCP, it naturally permits the input to be only partially specified
(rather than a fully formed image, for instance). 

Recall that all concurrent constraint programs $f$ describe
a function (a closure operator) that takes an input constraint $c$ to
an output $f(c)$.  Our basic idea is that instead of specifying a
fully executable cc program $f$, the programmer will supply a parametric version
$f_{\theta}$. How do the parameters show up in the code? Clearly, we
need to permit ask and tell constraints to contain parameters,
e.g. $X + \alpha \geq Y$ ($\alpha$ is the parameter). 

Additionally, we need to extend the notion of a constraint system to
admit a metric  $d: {\cal C} \rightarrow {\cal C} \rightarrow {\cal
  R}$ which takes pairs of constraints to a real value. Intuitively,
the metric measures how ``close'' the two constraints are, e.g.{} how
many valuations do the constraints differ on. 
The metric must satisfy certain obvious properties, such as symmetry,
equivalent constraints are $0$ apart from each other, and respect for
constraint equivalence. 
\begin{enumerate}
\item $d(c,c')=0$ if $c \vdash c', c' \vdash c$.
\item $d(b,c)=d(c,b)$
\item $d(b,c)=d(b,c')$ if $c \vdash c', c' \vdash c$.
\item {\em triangle inequality?}
\end{enumerate}

{\em Now we need to develop the algorithms that will take as input (1)
  the parametrized program, (b) the observations, (c) the loss
  function and output the value of the parameters that minimizes the
  loss over the observation set.}

Note that deep neural networks can be described as parametric CCP
programs. 

\newpage

\bibliographystyle{alpha}
%\bibliography{../../biblio}

\end{document}
