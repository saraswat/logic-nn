% This is LLNCS.DOC the documentation file of
% the LaTeX2e class from Springer-Verlag
% for Lbecture Notes in Computer Science, version 2.4
\documentclass{article} % For LaTeX2e
\usepackage{nips15submit_e,times}
\usepackage{hyperref}
\usepackage{url}
\usepackage[dvips]{graphicx}
\usepackage{xcolor}
%\usepackage{url}
\usepackage{colortbl}
\usepackage{multirow}


%\usepackage{amssymb}
%\newtheorem{definition}{Definition} % [section]
%\newtheorem{example}{Example} % [section]
\newcommand{\pivot}[1]{\mathbin{\, {#1} \,}}
\newcommand{\Pivot}[1]{\mathbin{\; {#1} \;}}
\newcommand{\Var}[0]{\mbox{\texttt{Var}}}
\newcommand{\tuple}[1]{\(\langle #1\rangle\)}
\def\withmath#1{\relax\ifmmode#1\else{$ #1 $}\fi}
\def\defeq{\withmath{\stackrel{d}{=}}}
\let\from=\leftarrow

\newcommand{\keywords}[1]{\par\addvspace\baselineskip
\noindent\keywordname\enspace\ignorespaces#1}
\usepackage{relsize}
\usepackage{turnstile}
\usepackage{amsmath}
\usepackage{url}

\usepackage{listings}

\usepackage{color}
%
\lstdefinelanguage{X10}%
  {morekeywords={abstract,break,case,catch,class,%
      const,continue,default,do,else,extends,false,final,%
      finally,for,goto,if,implements,import,instanceof,%
      interface,label,native,new,null,package,private,protected,%
      public,return,static,super,switch,synchronized,this,throw,%
      throws,transient,true,try,volatile,while,%
      async,atomic,when,foreach,ateach,finish,clocked,%
      type,here,%
      self,property,%
      proto,assert,%
      future,to,has,as,var,val,def,where,in,%
      value,or,await,current,any},%
   basicstyle=\normalfont\ttfamily,%\color{Red},%
   keywordstyle=\bf\ttfamily,%\color{OliveGreen},%
   commentstyle=\normalfont\ttfamily,%\color{Gray},%
   identifierstyle=\normalfont\ttfamily,%\color{Red},%
   stringstyle=\normalfont\ttfamily,%
   tabsize=4,%
   showstringspaces=false,%
   sensitive,%
   morecomment=[l]//,%
   morecomment=[s]{/*}{*/},%
   morestring=[b]",%
   morestring=[b]',%
   columns=fullflexible,%
   mathescape=false,%
   keepspaces=true,%
   showlines=false,%
   breaklines=true,%
   breakatwhitespace=true,%
   postbreak={},%
   %breakautoindent=true,%
   %breakindent=0pt,%
   %prebreak={},%
  }

\lstnewenvironment{xtenmath}
  {\lstset{language=X10,breaklines=false,captionpos=b,xleftmargin=2em,mathescape=true}}
  {}

\lstnewenvironment{xten}
  {\lstset{language=X10,breaklines=false,captionpos=b}} %,xleftmargin=2em
  {}

%{numbers=left, numberstyle=\tiny, stepnumber=2, numbersep=5pt
% numbers=left,
\lstset{language=x10,basicstyle=\ttfamily\small}
%\lstset{language=java,basicstyle=\ttfamily\small}


\usepackage[ruled]{algorithm} % [plain]
\usepackage[noend]{algorithmic} % [noend]
\renewcommand\algorithmiccomment[1]{// \textit{#1}} %


% For fancy end of line formatting.
\usepackage{microtype}


% For smaller font.
\usepackage{pslatex}


%\usepackage{xspace}
%\usepackage{yglabels}
%\usepackage{yglang}
%\usepackage{ygequation}
%\usepackage{graphicx}
%\usepackage{epstopdf}

\newcommand{\formalrule}[1]{\mbox{\textsc{\scriptsize #1}}}
\newcommand{\myrule}[2]{\textbf{Rule #1:} #2.}
%\newcommand{\myrule}[1]{\textsc{\codesmaller #1 Rule}}
\newcommand{\umyrule}[1]{\textbf{\underline{\textsc{\codesmaller #1 Rule}}}}


% \small \footnotesize \scriptsize \tiny
% \codesize and \scriptsize seem to do the same thing.
% \newcommand{\code}[1]{\texttt{\textup{\footnotesize #1}}}
% \newcommand{\code}[1]{\texttt{\textup{\codesize #1}}}
\newcommand{\normalcode}[1]{\texttt{\textup{#1}}}
\def\codesmaller{\small}
\newcommand{\myCOMMENT}[1]{\COMMENT{\small #1}}
\newcommand{\code}[1]{\texttt{\textup{\codesmaller #1}}}
% \newcommand{\code}[1]{\ifmmode{\mbox{\smaller\ttfamily{#1}}}
%                       \else{\smaller\ttfamily #1}\fi}
\newcommand{\smallcode}[1]{\texttt{\textup{\scriptsize #1}}}
%\newcommand{\myparagraph}[1]{\noindent\textit{\textbf{#1}}~} %\vspace{-1mm}\paragraph{#1}}

% See: \usepackage{bold-extra} if you want to do \textbf
\newcommand{\keyword}[1]{\code{#1}}

% For new, method invocation, and cast:

\newcommand{\Own}{{\it O}}
\newcommand{\Ifn}[1]{\ensuremath{I(#1)}}
\newcommand{\Ofn}[1]{\ensuremath{O(#1)}}
\newcommand{\Cooker}[1]{\ensuremath{{\kappa}(#1)}}
\newcommand{\Owner}[1]{\ensuremath{{\theta}(#1)}}
\newcommand{\Oprec}[0]{\ensuremath{\preceq_{\theta}}}
\newcommand{\Tprec}[0]{\ensuremath{\preceq^T}}
\newcommand{\TprecNotEqual}[0]{\ensuremath{\prec^T}}
\newcommand{\OprecNotEqual}[0]{\ensuremath{\prec_{\theta}}}
\newcommand{\IfnDelta}[1]{\ensuremath{I_\Delta(#1)}}
\newcommand\Abs[1]{\ensuremath{\left\lvert#1\right\rvert}}
\newcommand{\erase}[1]{\ensuremath{\Abs{#1}}}

\newcommand{\Gdash}[0]{\ensuremath{\Gamma \vdash }}
\newcommand{\reducesto}[0]{\rightsquigarrow}
\newcommand{\reduce}[0]{\rightsquigarrow}
\newcommand{\preduce}[0]{\rightsquigarrow_{\pi}}
\newcommand{\plreduce}[0]{\rightsquigarrow_{\pi[\ol{\hl}]}}
\newcommand{\ureduce}[1]{\stackrel{{#1}}{\dashrightarrow}} %\rightsquigarrow\stackrel{a}
\newcommand{\ptreduce}[0]{\rightsquigarrow_{\pi'}}


%% For typesetting theorems and some math symbols (\rightsquigarrow).
\usepackage{amssymb}
%\usepackage{amsthm}

\usepackage{color}
\definecolor{light}{gray}{.75}


\newcommand{\todo}[1]{\textbf{[[#1]]}}
%% To disable, just uncomment this line:
%\renewcommand{\todo}[1]{\relax}

%% Additional todo commands:
\newcommand{\TODO}[1]{\todo{TODO: #1}}

\newcommand\xX[1]{$\textsuperscript{\textit{\text{#1}}}$}


\newcommand{\ol}[1]{\overline{#1}}
\newcommand{\nounderline}[1]{{#1}}


%% Commands used to typeset the FIGJ type system.
\newcommand{\typerule}[2]{
\begin{array}{c}
  #1 \\
\hline
  #2
\end{array}}
\newcommand{\typeax}[1]{
\begin{array}{c}
   \\
\hline
  #1
\end{array}}


%% Commands used to typeset the FOIGJ type system.
\newcommand{\inside}{\prec}
\newcommand{\visible}{{\it visible}}
\newcommand{\placeholderowners}{{\it placeholderowners}}
\newcommand{\nullexpression}{{\tt null}}
\newcommand{\errorexpression}{{\tt error}}
\newcommand{\locations}{{\it locations}} %% \mathop{\mathit{locations}}}
\newcommand{\xo}{{\tt X^O}}
\newcommand{\no}{{\tt N^O}}
\newcommand{\co}{{\tt C^O}}
\newcommand{\I}{\it I}


\newcommand\mynewcommand[2]{\newcommand{#1}{#2\xspace}}


\mynewcommand{\unknown}{\code{?}}
\newcommand{\initsep}[0]{;} % tried \| and \dagger
\newcommand{\initsets}[2]{\lb#1\initsep#2\rb}
\newcommand{\myinit}[2]{\S{}\lb#1\initsep#2\rb}
\mynewcommand{\mycooked}{\S{}\lb\initsep\rb}
\newcommand{\valt}[2]{\code{val}~#1~=~#2;}
\newcommand{\acct}[2]{\code{acc}~#1~=~#2;}

\newcommand{\lt}{\code{<}}%{\mathop{\textrm{\tt <}}}
\newcommand{\gt}{\code{>}}%{\mathop{\textrm{\tt >}}}

\mynewcommand{\this}{\keyword{this}}
\mynewcommand{\Object}{\code{Object}}
\mynewcommand{\const}{\keyword{const}} %C++ keyword
\mynewcommand{\mutable}{\keyword{mutable}} %C++ keyword
\mynewcommand{\romaybe}{\keyword{romaybe}} %Javari keyword

%% Define the behaviour of the theorem package.
%% Use http://math.ucsd.edu/~jeggers/latex/amsthdoc.pdf for reference.

\newtheorem{theorem}{Theorem}[section]
\newtheorem{definition}[theorem]{Definition}
%\newtheorem{lemma}[theorem]{Lemma}
%\newtheorem{corollary}[theorem]{Corollary}
%\newtheorem{fact}[theorem]{Fact}
\newtheorem{proposition}[theorem]{Proposition}
%\newtheorem{convention}[theorem]{Convention}
\newtheorem{example}[theorem]{Example}
%\newtheorem{remark}[theorem]{Remark}
\def\withmmode#1{\relax\ifmmode#1\else{$#1$}\fi}
\def\alt{\withmmode{\;{\tt\char`\|}\;}}

\mynewcommand{\IP}{\code{I}}   % formal type parameter
\mynewcommand{\JP}{\code{J}}   % formal type parameter (for soundness proofs)


\mynewcommand{\Iparam}{Immutability parameter}
\mynewcommand{\iparam}{immutability parameter}
\mynewcommand{\iparams}{immutability parameters}
\mynewcommand{\Iparams}{Immutability parameters}
\mynewcommand{\Iarg}{Immutability argument}
\mynewcommand{\iarg}{immutability argument}
\mynewcommand{\iargs}{immutability arguments}
\mynewcommand{\Iargs}{Immutability arguments}
\mynewcommand{\ReadOnly}{\code{ReadOnly}}
\mynewcommand{\WriteOnly}{\code{WriteOnly}}
\mynewcommand{\None}{\code{None}}
\mynewcommand{\Mutable}{\code{Mutable}}
\mynewcommand{\Immut}{\code{Immut}}
\mynewcommand{\Raw}{\code{Raw}}


\mynewcommand{\This}{\code{This}}
\mynewcommand{\World}{\code{World}}


% Our annotations
\mynewcommand{\OMutable}{\code{@OMutable}}
\mynewcommand{\OI}{\code{@OI}}


\mynewcommand{\InVariantAnnot}{\code{@InVariant}}


\newcommand{\func}[1]{\text{\textnormal{\textit{\codesmaller #1}}}}


\mynewcommand{\st}{\ensuremath{\mathrel{{\leq}}}} %{\mathop{\textrm{\tt <:}}}
\mynewcommand{\notst}{\mathrel{\st\hspace{-1.5ex}\rule[-.25em]{.4pt}{1em}~}}
\mynewcommand{\tl}{\ensuremath{\triangleleft}}
\mynewcommand{\gap}{~ ~ ~ ~ ~ ~}


\newcommand{\RULE}[1]{\textsc{\scriptsize{}#1}} %\RULEhape\scriptsize}


\mynewcommand{\DA}{\texttt{DA}}
\mynewcommand{\ok}{\texttt{OK}}
\mynewcommand{\OK}{\texttt{OK}}
\mynewcommand{\IN}{\texttt{IN}}
\mynewcommand{\subterm}{\func{subterm}}
\mynewcommand{\TP}{\func{TP}} % function that returns type parameters in a type
\mynewcommand{\CT}{\func{CT}} % class table
\mynewcommand{\mtype}{\func{mtype}}
\mynewcommand{\mbody}{\func{mbody}}
\mynewcommand{\mmodifier}{\func{mmodifier}}
\mynewcommand{\fmodifier}{\func{fmodifier}}
\mynewcommand{\fields}{\func{fields}}
\mynewcommand{\cooked}{\func{cooked}}


\mynewcommand{\facc}{\func{acc}}
\mynewcommand{\fclocked}{\func{clocked}}

\mynewcommand{\bound}{\func{bound}_\Delta}
\mynewcommand{\substitute}{\func{substitute}}
\mynewcommand{\ftype}{\func{ftype}}
\mynewcommand{\mguard}{\func{mguard}}
\mynewcommand{\isTransitive}{\func{isTransitive}}
\DeclareMathOperator{\dom}{dom}


\newcommand{\async}[1]{\code{async}~#1}
\newcommand{\asynct}[2]{\code{async(}{\tt #1})\lb#2\rb}
\newcommand{\finisht}[2]{\code{finish(}{\tt #1})\lb#2\rb}
\newcommand{\blockt}[2]{\code{(}{\tt #1}\code{)}\lb#2\rb}
\newcommand{\finishasynct}[2]{\code{[finish | async]}({\tt #1})\lb#2\rb}
\newcommand{\clockedfinishasync}[1]{\code{(async | [clocked] finish)}~{\tt #1}}
\newcommand{\finishasync}[1]{\code{[finish | async]}~#1}
\newcommand{\finish}[1]{\code{finish~#1}}
\newcommand{\clocked}[0]{\code{clocked}}
\newcommand{\pclocked}[0]{\code{[clocked]}}
\newcommand{\xadvance}[0]{\code{advance;}}
\newcommand{\acc}[2]{\code{acc}~ #1=#2;}

% In the syntax: \hI or ReadOnly or Mutable or Immut



\mynewcommand{\hnull}{\code{null}}
\mynewcommand{\htrue}{\code{true}}
\mynewcommand{\hfalse}{\code{false}}

\mynewcommand{\hUnused}{\code{\_}}
\mynewcommand{\hA}{\code{A}} % inVariant definition
\mynewcommand{\hB}{\code{B}} % inVariant definition
\mynewcommand{\hC}{\code{C}} % class
\mynewcommand{\hD}{\code{D}} % class
\mynewcommand{\hF}{\code{F}} % field declaration
\mynewcommand{\hG}{\code{G}}
\mynewcommand{\hI}{\code{I}} % iparam
\mynewcommand{\hJ}{\code{J}}
\mynewcommand{\hL}{\code{L}} % class decl
\mynewcommand{\hM}{\code{M}} % Method decl
\mynewcommand{\hN}{\code{N}} % Non-variable type
\mynewcommand{\hO}{\code{O}}
\mynewcommand{\hP}{\code{P}}
\mynewcommand{\hR}{\code{R}}
\mynewcommand{\hS}{\code{S}}
\mynewcommand{\hT}{\code{T}} % types (vars or non vars)
\mynewcommand{\hU}{\code{U}} % types (vars or non vars)
\mynewcommand{\hV}{\code{V}} % closed types
\mynewcommand{\hX}{\code{X}} % vars
\mynewcommand{\hY}{\code{Y}} % vars
\mynewcommand{\hZ}{\code{Z}} % inVariant definition


\mynewcommand{\ha}{\code{a}}
\mynewcommand{\hc}{\code{c}} % cooker
\mynewcommand{\hd}{\code{d}}
\mynewcommand{\hm}{\code{m}} % method
\mynewcommand{\he}{\code{e}} % expression
\mynewcommand{\hf}{\code{f}} % field
\mynewcommand{\hg}{\code{g}}
\mynewcommand{\hl}{\code{l}} % location in the store
\mynewcommand{\ho}{\code{o}}
\mynewcommand{\hp}{\code{p}} % cooker
\mynewcommand{\hq}{\code{q}}
\mynewcommand{\hr}{\code{r}}
\mynewcommand{\hv}{\code{v}} % value
\mynewcommand{\hw}{\code{w}} % value
\mynewcommand{\hx}{\code{x}} % method parameter
\mynewcommand{\hy}{\code{y}} % field
\mynewcommand{\hz}{\code{z}} % method parameter

\mynewcommand{\lroot}{\code{l}_\top} % root
\mynewcommand{\lthis}{\code{l}_\smallcode{this}} % this
\newcommand{\hparen}[1]{\code{(}#1\code{)}}
\newcommand{\hgn}[1]{\lt#1\gt} % type parameters and generic method parameters
\newcommand{\hadvance}[0]{\code{advance}}
\mynewcommand{\hasync}{\code{async}}
\mynewcommand{\hswitch}{\code{switch}}
\mynewcommand{\hAcc}{\code{Acc}}
\mynewcommand{\hfinish}{\code{finish}}
\mynewcommand{\hreceiver}{\code{receiver}} % receiver for new
\mynewcommand{\hSW}{\code{SW}}
\mynewcommand{\hAW}{\code{AW}}
\mynewcommand{\hObject}{\code{Object}}

\mynewcommand{\hdef}{\code{def}}
\mynewcommand{\hfor}{\code{for}}
\mynewcommand{\hvar}{\code{var}}
\mynewcommand{\hin}{\code{in}}
\mynewcommand{\hPoint}{\code{Point}}
\mynewcommand{\hand}{\code{~and~}}
\mynewcommand{\hor}{\code{~or~}}
\mynewcommand{\hthis}{\code{this}} % this
\mynewcommand{\hclass}{\code{class}}
\mynewcommand{\hreturn}{\code{return}}
\mynewcommand{\hhnew}{\code{new}}
\mynewcommand{\hsub}{\code{/}} % substitute (reduction rules)
%\newcommand{\Ofn}[1]{\ensuremath{O(#1)}}
\mynewcommand{\nonescaping}{\code{nonescaping}}
\mynewcommand{\hescaping}{\code{escaping}}
\mynewcommand{\hextends}{\code{extends}}
\newcommand{\hnew}[1]{\code{new}~#1}
\newcommand{\hval}[3]{\code{val}~#1~=~#2;#3}
\newcommand{\hclocked}[0]{\code{clocked}}
\newcommand{\hClockedAcc}[0]{\code{ClockedAcc}}
\newcommand{\hvalues}[0]{\code{values}}

\newcommand{\PREV}[1]{\withmmode{\mathtt{prev}~{#1}}}
\newcommand{\ift}[2]{\withmmode{\mathtt{if}~{#1}~\mathtt{then}~{#2}}}
\newcommand{\ife}[2]{\withmmode{\mathtt{if}~{#1}~\mathtt{else}~{#2}}}
\newcommand{\some}[2]{\withmmode{\mathtt{some}~{#1}\,\mathtt{in}\,~{#2}}}
\newcommand{\all}[2]{\withmmode{\mathtt{all}~{#1}\,\mathtt{in}\,{#2}}}
\newcommand{\hence}[1]{\withmmode{\mathtt{hence}~{#1}}}
\newcommand{\hitherto}[1]{\withmmode{\mathtt{hitherto}~{#1}}}
\newcommand{\AND}[2]{\withmmode{{#1}~\mathtt{and}~{#2}}}
\newcommand{\OR}[2]{\withmmode{{#1}~\mathtt{or}~{#2}}}
\newcommand{\MU}[2]{\withmmode{\mathtt{mu}~{#1}\ \mathtt{in}\ {#2}}}
\newcommand{\past}[2]{\withmmode{\mathtt{past}^{#1}\,{#2}}}
\newcommand{\TIME}[2]{\withmmode{\mathtt{time}^{#1}\,{#2}}}

\newcommand{\proves}[3]{\withmmode{\dststile{#3}{#1,#2}}}
\newcommand{\evolves}[3]{\withmmode{\longrightarrow^{#1,#2}_{#3}}}
\newcommand{\starevolves}[3]{\withmmode{\stackrel{\star}{\longrightarrow}^{#1,#2}_{#3}}}
\newcommand{\steps}[2]{\withmmode{\leadsto^{#1,#2}}}
\mynewcommand\xth{\xX{th}}
\mynewcommand\xrd{\xX{rd}}
\mynewcommand\xnd{\xX{nd}}
\mynewcommand\xst{\xX{st}}
\mynewcommand\ith{$i$\xth}
\mynewcommand\jth{$j$\xth}


%\mynewcommand{\emptyline}{\vspace{\baselineskip}}
\mynewcommand{\myindent}{~~}


% Add line between figure and text
\makeatletter
\def\topfigrule{\kern3\p@ \hrule \kern -3.4\p@} % the \hrule is .4pt high
\def\botfigrule{\kern-3\p@ \hrule \kern 2.6\p@} % the \hrule is .4pt high
\def\dblfigrule{\kern3\p@ \hrule \kern -3.4\p@} % the \hrule is .4pt high
\makeatother

\setlength{\textfloatsep}{.75\textfloatsep}


% Remove line between figure and its caption.  (The line is prettier, and
% it also saves a couple column-inches.)
\makeatletter
%\@setflag \@caprule = \@false
\makeatother


% http://www.tex.ac.uk/cgi-bin/texfaq2html?label=bold-extras
\usepackage{bold-extra}


% Left and right curly braces in tt font
\newcommand{\ttlcb}{\texttt{\char "7B}}
\newcommand{\ttrcb}{\texttt{\char "7D}}
\newcommand{\lb}{\ttlcb}
\newcommand{\rb}{\ttrcb}
\newcommand{\cc}{{\sf CC}}
\newcommand{\TDCC}{{\sf TDCC}}


\setlength{\leftmargini}{.75\leftmargini}
\def\comment#1{\typeout#1}

\def\from#1\infer#2{{{\textstyle #1}\over{\textstyle #2}}}
\def\rname#1\from#2\infer#3{{{\textstyle #2}\over{\textstyle #3}}{\ \textstyle(#1)}}


\begin{document}
%\bibliographystyle{acmtrans}

\long\def\comment#1{}
\def\mtimes{}
\def\LL#1{#1}
\title{Notes on differentiable logic \\
{\small Logic through vector calculus} \\
{\small ***DRAFT v0.05 ***}}

\author{Vijay Saraswat \\
IBM T.J. Watson Research Center\\
1101 Kitchawan Road\\
Yorktown Heights, NY 10598 \\
\texttt{vijay@saraswat.org} \\
\And
Radha Jagadeesan \\
De Paul University \\
243 S. Wabash Avenue \\
Chicago, IL 60604 \\
\texttt{rjagadeesan@cs.depaul.edu}
}

\newcommand{\fix}{\marginpar{FIX}}
\newcommand{\new}{\marginpar{NEW}}

\nipsfinalcopy % Uncomment for camera-ready version

\maketitle

\begin{abstract}
First-order logic has crisp notions of individuation and predication: individuals are taken  to be fully formed objects from some underlying set $U$ which typically has no internal structure, and $k$-ary predicates are taken to be $k$-ary maps from $U$ to $B=\{0,1\}$ that specify whether a particular tuple is in the relation or not. 

Recent years have seen some remarkable successes in the application of machine learning techniques to a variety of applications in speech, vision and natural language understanding. Typically, these successes are based on the use of continuous (non-convex) optimization techniques (such as gradient descent). The need to account for logical structure has led to the development of several {\em ad hoc} techniques based on embedding individuals into $n$-dimensional vectors (``distributional representations'', \cite{unified-arch-nlp,mikolov-word-vec,baroni2014don}), and treating predicates as tensors \cite{coecke-2010,grefenstette-2013,socher-ntn-2013,bishan-iclr15,rocktaschel-2015,order-embedding-kiros,TransIE}, without a proper understanding of how these techniques relate to logic. 

%% TODO: 1-Lipschitz vs continuous.
In these notes we give an account of a precise embedding of first-order logic (over finitary structures) into vector spaces over $\Re$, via the ``Fourier expansion'' of Boolean functions as multilinear polynomials. This immediately permits a continuous embedding of predicates into $[0,1]^n \rightarrow [0,1]$ that enables techniques such as gradient descent to be used directly on such formulas.

%\keywords{machine learning, concurrent constraint programming}
\end{abstract}

\def\Or{\vee}
\def\And{\wedge}
\def\Arrow{\rightarrow}
\def\Xor{\;\mbox{xor}\;}
\def\Ind{\;\mbox{Ind}}

\section{Introduction}
The {\em supervised machine learning} approach is applicable in settings where the programmer is concerned with specifying a function that must work accurately even in the presence of significant amounts of noise in the input. For instance, suppose the programmer must write code that labels an arbitrary input image with a tag (``dog with hat'') that best characterizes the image. The space of variations in the input is generally so vast and potentially so ill-understood mathematically that it may simply not be possible for the user to programmatically specify all the logic for the function.

Instead we desire an approach whereby (portions of the) logic may be {\em learnt} by automatic techniques, given a (potentially very) large collection of observations (input/output pairs) of the given function. 

Concretely, we may describe the problem as follows. Instead of providing a fully realized, executable function $f$, the programmer specifies a space $\cal F$ of possible functions, obtained by varying the parameters $\theta$ of some parametric function $f_{\theta}$.  Also available is the set of {\em observations} $(\bar{x}_i, \bar{y}_i)$ (for $i\in I$) specifying pairs of input values with their associated output values, and a {\em loss function} $L$ which can take two values $\bar{y},\bar{y}'$ and return a real value  $L(\bar{y},\bar{y}')$ which measures how far one value is from the other. The machine learning problem is now to determine a specific value $\hat{\theta}$ for the parameters which minimizes $$\Sigma_{i \in I} L(f_\theta(\bar{x}_i), \bar{y}_i)$$

Many algorithmic techniques are available, under various conditions, to address this problem. For instance, in situations where $f_{\theta}$ represents a differentiable function of its parameters $\theta$, one may use stochastic gradient descent to find a minima. Under other conditions, the Expectation Maximization algorithm may be used.

A very powerful instance of this general picture is obtained with (feed forward) {\em deep neural networks} (see, e.g., \cite{deep-learning-nature-2015}).  In this setting, the parametric function $f_{\theta}$ is given by $k$ layers of compute elements (called ``neurons''), an element in layer $i$ connected to some (possibly all) elements in the preceding layer. Each neuron is associated with a set of weights $W$ (its parameters) used to linearly sum its inputs. (There is also usually a bias value.) Some non-linear (but piece-wise differentiable) thresholding function (e.g. sigmoid, $\lambda t.\,(1/(1+e^{-t}))$) is used to determine the output of this element. One can think of the entire network as specified by a $k$-nested functional term, involving summations and pointwise applications of the thresholding function. It has been shown that any arbitrary non-linear (real-valued) function can be approximated arbitrarily closely with sufficiently many parameters, and sufficient training input. 

What has made the machine learning problem of great interest in recent years is the availability of large amounts of labeled training data, and vast amounts of (CPU, GPU) computation that can be brought to bear to train the given function. A better understanding of how training algorithms such as asynchronous stochastic gradient networks  \cite{distbelief} can be made to work for deep networks has brought about a startling increase in accuracy in many applications, and a significant increase in the range of applications amenable to these techniques.

\subsection{Differentiable programming languages}
A significant drawback of deep neural networks is their {\em   opacity}. In general, it is not possible to give a meaningful answer to the question of {\em why} the learnt program does what it does. At hand is simply the value of the (potentially billions of) parameters, poor material from which to construct causally coherent explanations accessible to a human observer.


We are interested in developing the foundations of a powerful framework for machine learning and reasoning which gives the programmer flexibility in expressing significant amount of prior knowledge in the struture of the space $\cal F$. We wish to use the full power of a general-purpose programming framework, allied with machine learning. Conceptually, the programmer may write programs in a high-level language while leaving ``holes'' (also called {\em   sketches}, \cite{solar-lezama:asplos06}) to be filled in by an appropriate automated assistant, such as a supervised machine learning algorithm, or a constraint solver. For instance, a program may leave the definition of a procedure {\em q?} unspecified, merely leaving in place an invocation of the procedure on particular arguments (e.g.{} {\em q?(X,3,true)}). The task of the solver is to determine the definition of the procedure so that the overall program exhibits the desired observations (approximately).   

While considerable progress has been made in the last decade on the use of constraint solvers for sketching (see, e.g., \cite{armando}), the use of machine learning for sketching is an unexplored area. In particular, it calls for developing a ``differentiable'' programming language in which programs are embedded in a continuous space, may have user-specified parameters, and are end-to-end differentiable with respect to these parameters.  

Of particular interest to us are programming languages based on logic -- such as definite clause logic programming \cite{Kowalski76} and concurrent constraint programming \cite{CCP}, including probabilistic versions thereof \cite{muggleton:srl07,Gupta97b,cussens}. Such frameworks carry dual advantages -- they permit the direct expression of knowledge about the real world (via first-order formalizations), and they offer the full power of programming languages for algorithmic tasks. 

The focus of this note is to establish the foundations for a continuous embedding of logic. Future version of this note / subsequent papers will develop machine learning applications and implementations. 

\subsection{Related work}

%% Also describe recent general end-to-end differentiable approaches. Neural Turing Machines.

%% Work on open IE and matrix completion as a way of learning new predicates.

%% Work by Bishan Yang, and Rajarshi Das and others on doing inference in knowledge graph through neural network techniques.

%% Work on Neural Theorem Proving

%% Work by Grefenstette and Cock on using tensors for first order formulas.
\section{Basic Idea}
We develop a continuous embedding for first-order logic, building on the ideas of \cite{boolean-function}. Let us quickly sketch the construction.

Let $B=\{0,1\}$, $I=[0,1]$, $U=B^n$ (the space of $n$-vectors over $B$) and $V=I^n$. The universe of individuals is taken to be $U$, predicates are thought of as {\em multilinear polynomials} from $U^k$ to $B$, and logical functions are taken as maps from $U^k$ to $U$. The polynomial representation (also called the {\em Fourier expansion}) provides the basis for an embedding of  predicates into the space of continuous functions $V^k \rightarrow I$ and an embedding of logical functions into continuous functions $V\rightarrow V$.   

Note that the set of all Boolean functions in $n$ variables forms a vector space over $\Re$. The elementary monomials $\Pi_{i\in A} x_i$ (with $A \subseteq \{1,\ldots, n\}$) form a basis. (The identity $x\mtimes x=x$ can be used to reduce all monomials to this form.) 

We now proceed step by step. 
\subsection{Propositional logic}\label{section:prop-logic}
Let the underlying base domain be $B$, with $0$ interpreted as false and $1$ as true. Define $\bar{a}$ as $1-a$. Define:

\begin{align*}
  \LL{\neg a} &= \bar{\LL{a}}\\
  \LL{a \And b} &= \LL{a}\mtimes \LL{b}
\end{align*}

The definition of the other logical operators follows:
  \begin{align*}
  \LL{a\Or b} &= \overline{\bar{a}\bar{b}} \\
  \LL{a \Arrow b} &= \overline{\LL{a}\mtimes\bar{\LL{b}}}\\
  \LL{a \Xor b} &= \LL{a} + \LL{b} - 2\mtimes\LL{a}\mtimes\LL{b}=(\LL{a}-\LL{b})^2\\
  \LL{(a = b)} &=\LL{a}\mtimes \LL{b} + \bar{\LL{a}}\bar{\LL{b}} =\overline{(\LL{a}-\LL{b})^2}
  \end{align*}

The indicator function $\Ind_a(x)$ which for $a,x \in B$ returns
$1$ if $a$ equals $x$ and $0$ otherwise, is just $\_=\_$.

\subsection{Representing first-order logic}

\subsubsection{Representing the domain of discourse}
For now we shall consider only finite domains (the finiteness assumption of the underlying domain is standard practice in machine learning).

We shall now make an assumption fundamental to this vector space approach to logic: individuals can be associated with vectors over \(B^n\). That is, we assume that the space of individuals carries ``internal'' structure: an individual is, intrinsically, a linear combination of \(n\) unit vectors in some space, and the predicates of interest to us respect this linear structure. Note that the ``classical'' domain of discourse for logic -- consisting of a set of \(n\) individuals with no ``internal'' connection to each other -- can be accommodated in the current setting by considering a space with \(n\) dimensions, and associating each individual with a unit vector along a given dimension. Thus the individuals are ``linearly independent'' of each other (orthogonal). This use of ``one-hot'' representations comes at a price -- $n$ bits are used to represent a set of size \(n\), rather than a set of size \(2^n\). %\footnote{Of course, a ``sparse'' representation of the vector could be used, each vector being represented by the index of the unique element that is \(1\) -- this is a representation of size \(\log_2(n)\).} 

The justification for such an assumption  comes from the ``associational'' theories of Robert Firth (``you shall know the meaning of words by the company they keep''), \cite{firth-57}. In natural language processing, such a ``vector representation'' of words (including words such as {\sc London} and {\sc Paris}) has met with some success.  Various mechanisms have been devised (e.g.{} \cite{word-to-vec}) using associational techniques to associate dense vectors with objects. Determining the dimensionality of this vector space now becomes a critical task in this approach to knowledge representation. Empirically, \(n=300\) and \(n=600\) have been used; likely these numbers are low for general knowledge representation tasks. 

%We shall assume an external mechanism has been used to identify the individuals in the domain of interest to us (e.g.{} the characters in some novel), based on empirically available data. Further, this may have been done using ``distributional'' techniques, so that the individuals are in fact represented as (dense) vectors in some high dimensional space $U$ \cite{word-to-vec}. This linear representation automatically comes with a ``built-in'' similarity relation $\sim$ (the cosine distance between the vectors). The value of $n$ will depend on the amount of data available and the number of individuals; $n=300$ and $n=600$ have been used to create vectors for words in various corpora. Determining the right value of \(n\) is a critical task in this approach to knowledge representation.


We can extend the technique of indicator functions discussed above to $n$-vectors. By $\Ind_a(x)$ (for $a,x$ $n$-vectors over \(B\)) we will mean \(\Pi_{i   \in 1\ldots n} \Ind_{a_i}(x_i)\).  Therefore, any function \(f:B^n \rightarrow \Re\) can be represented as a multilinear polynomial (its ``Fourier expansion'') \cite{boolean-function}: 
\begin{equation}\label{eqn:fourier-expansion}
\Sigma_{a \in \{0,1\}^n} f(a)\Ind_{a}(x)
\end{equation}


Alternatively, we can use a different basis, the monomials \cite{PB-optimization}. Every
function $f:\{0,1\}^n \rightarrow \Re$ can be uniquely represented as a multilinear polynomial: 
$$ c_0 + \Sigma_{k \in 1\ldots m} (c_k\; \Pi_{i \in A_k} x_i)$$
\noindent for real coefficients $c_0, c_1, \ldots c_m$ and non-empty subsets $A_1, \ldots, A_m$ of $N = \{1, \ldots, n\}$. 

\begin{example}
Let us work out $\mbox{and2}: \{0,1\}^2 \rightarrow \Re$ (a unary function that returns 1 iff both components are 1):

\begin{align*}
and2(x)  &= 1\mtimes \Ind_{(1,1)}(x) + 0\mtimes\Ind_{(1,0)}(x) + 0\mtimes\Ind_{(0,1)}(x) + 0\mtimes\Ind_{(0,0)}(x)\\
&= \Ind_{(1,1)}(x) \\
&= (1 \mtimes x_1+0\mtimes(1-x_1))\mtimes(1 \mtimes x_2+0\mtimes(1-x_2))\\
  &=x_1\mtimes x_2 
\end{align*}
\end{example}

\subsubsection{Representing first-order predicates}
We can use the technique of indicator functions to represent a $k$-ary predicate $p$ as a multilinear polynomial over $kn$ variables. 
$$p(x_1,\ldots, x_k)= \Sigma_{a_1,\ldots,a_k \in \{0,1\}^n} p(a_1, \ldots, a_k) \mtimes \Pi_{i \in 1\ldots k} \Ind_{a_i}(x_i)$$

\noindent Note that this technique can be used to represent the equality predicate: Here $p(a,b)=1$ if $a=b$, and $0$ otherwise.

% TODO: vj Check. Why do we need this?
A $k$-ary function $f:U^k \rightarrow U$ is represented by $n$ polynomials, one for each dimension.

The Boolean operations are defined as in Section~\ref{section:prop-logic}. Since the domain is finite, existential quantification is represented by a finite disjunction, and universal quantification by a finite conjunction. 

Thus, any first-order formula (with equality) in a logical vocabulary \({\cal L}\) can be represented as multilinear polynomial over \(B\). 

\subsection{Tensor representation}
A multilinear function $U^k \rightarrow \Re$ can be represented directly as a (covariant) tensor $T^k(U)$ \cite[Chapter 8]{lee-book-2000}. The tensor maps a tuple \tuple{u_1,\ldots, u_k} to $1$ if the tuple is in the relation and to $0$ otherwise. Function application is implemented through tensor contraction. Note that in practice such a tensor is implemented extensionally merely as a \(k\)-dimensional array with \(n\) elements in each dimension (thus requiring \(n^k\) numbers). 

A similar observation is made in \cite{grefenstette-2013} (though the underlying connection with multilinear polynomials is not).

For more details, please see Appendix~\ref{sec:tensor} where a relatively self-contained background is provided for tensors, and examples worked out. 

\subsection{Continuous embedding}
We now take the all important step of passing to the continuous version of the above representation. 
The continuous representation of a first-order formula $p$, $p^c$ is given by the above polynomial, with
variables ranging over $[0,1]$. It is easy to see that the result will also be in $[0,1]$. (The result is the weighted sum of $p(a_1, \ldots, 
  a_k)$, with the weight supplied by the distance between $(x_1,
  \ldots, x_k)$ and $(a_1, \ldots, a_k)$ using $\Ind_{a_i}(x_i)$. The
  weights sum to $1$, as can be shown by an inductive argument,
  leveraging the fact that $\Ind_{0}(x) + \Ind_{1}(x)=(1x+0(1-x))+(0x+1(1-x))=1$.)

Similarly for $f^c$. 

The intuition is that $[0,1]^n$ represents a continuous smearing of the space $\{0,1\}^n$ of ``ideal'' individuals. The polynomial representation of a relation faithfully represents the behavior of the original (discrete) relation on the ideal points, and represents the behavior on intermediate (``smeared'') points as a linear combination of the behavior on ideal points.
% TODO: Should be able to formulate using probabilities and expectations.

The rules for differentiation of these polynomials wrt a variable are standard. 

\begin{example}
Assume the underlying domain is expressible as 2-dimensional vectors. Let {\em bill} be represented by \((1, 0)\), {\em hilary} by \((0,1)\), and the extension of the predicate {\em love/2} be given by
\begin{lstlisting}
love(bill,hillary).
love(hilary,bill).
love(hilary,hilary).
\end{lstlisting}
The representation of {\em love(x,y)} is:
\begin{align*}
&\Ind_{(1,0)}(x)\Ind_{(0,1)}(y) + \Ind_{(0,1)}(x)\Ind_{(1,0)}(y) + \Ind_{(0,1)}(x)\Ind_{(0,1)}(y)  \\
&= x_1\bar{x_2}\bar{y_1}y_2 + \bar{x_1}x_2y_1\bar{y_2} + \bar{x_1}x_2\bar{y_1}y_2
\end{align*}
For \(((1,0),(0,1))\) the polynomial evaluates to \(1\); for \(((1,0),(1,0))\) it evaluates to \(0\). 

For someone \((0.9,0.2)\) more like {\em bill} than {\em hilary}, the degree to which they love
{\em bill} is given by: 
\(((0.9)(0.8)(0)(0) + (0.1)(0.2)(1)(1) + (0.1)(0.2)(0)(0)=0.02\).
%$((0.9)(0.8)(1)(1) + (0.1)(0.2)(0)(0) + (0.1)(0.2)(1)(1)=(0.72)+(0.02)=0.74$.

For someone (\((0.5,0.5)\)) with characteristics of both {\em bill} and {\em hilary} we get the degree to which they love {\em hilary}  is given by
%$(0.5)(0.5)(0.5)(0.5)+(0.5)(0.5)(0.5)(0.5)+(0.5)(0.5)(0.5)(0.5)=0.1875$.
\((0.5)(0.5)(1)(1)+(0.5)(0.5)(0)(0)+(0.5)(0.5)(1)(1)=0.5$, and the degree to which they love {\em bill} is given by $(0.5)(0.5)(0)(0) + (0.5)(0.5)(1)(1) + (0.5)(0.5)(0)(0)=0.25\).

\end{example}

\subsection{Expressiveness}
% TODO: motivate and write up more extensively.
% Define 1-Lipschitz. Motivate why they are interesting in this context.
We note that by adding max and step functions we can represent any $1$-Lipschitz function in the following sense.
%This monomials as discussed above, together with max and step functions

Consider the metric \(d(X,Y) = max \{\lvert x_i - y_i \rvert \mid i\in 1\ldots n \}\) over \(I^n\). For a 
rational number \(r\) s.t. \(0 \leq r \leq 1\) and \(Q\) a vector of rational numbers, let \(step_{r,Q}:I^n \rightarrow \Re\) be defined as \(max(0, r-d(X,Q))\). This is the ``step'' function that is \(r\) at \(X\), then drops off to zero at the maximum possible rate. 

\begin{definition}
  Let \(\cal F\) be the functions generated thus: any multilinear function is in \(\cal F\); if \(f,g \in {\cal F}\), then so is \(max(f,g)\); \(step_{r,Q} \in {\cal F}\) for rational \(r \in I\) s.t. \(0 \leq r \leq 1\) and vector \(Q\) of rationals.
\end{definition}

\begin{theorem}
  For any \(1\)-Lipschitz function \(g:I^n \rightarrow \Re\) and \(\epsilon\) there exists an \(f \in {\cal F}\) s.t. for all \(X\), \(\lvert g(X)-f(X) \rvert \lt \epsilon\).
\end{theorem}
In proof, consider the set \(\{ step_{r,Q} \alt f(Q) \ge r \}\). Closing this set under finite \(max\) yields a increasing sequence of functions that converges uniformly to \(f\).

(Note: this proof does not depend on monomials.)

\section{Applications to Machine Learning}


We outline some directions that open up as a result of our approach.

\subsection{Learning arbitrary definite clauses}
The representation of formulas as polynomials makes it possible to use symbolic approaches to answer some logical questions. Here we discuss the {\em rule extraction} task of \cite[Algorithm 1]{bishan-iclr15}.

The basic approach in the paper is as follows. Entities are represented by $n$-vectors (as we discuss). A binary relation $p$ is represented as a matrix $\LL{p}$, with a predication $p(x,y)$ evaluated as $\LL{x}^t \LL{p}\LL{y}$.\footnote{\cite{bishan-iclr15} also discusses a form where predicates are represented as vectors; however this does not yield good results.} Their approach learns {\em closed path} definite clauses; these are clauses of the form:
$$ p_1(X_1, X_2), p_2(X_2,X_3), \ldots, p_n(X_n, X_{n+1}) \rightarrow p(X_1, X_{n+1})$$
Here $p_i$ are known binary relations (trained matrices are available for them); this set is closed under inversion, since the inverse of a relation can be obtain by matrix operations. The basic idea is to find ``high probability paths'' in the knowledge graph (entities are types of nodes, predicates are binary edges), and compute the matrix representing the relation between the starting vertex $a$ and the final vertex $b$ by composing the matrices of the relations along the path. This is then compared with the existing relation labeling the edge $(a,b)$ using the $L_2$ norm; the rule is accepted if the norm is below a certain threshold. An example of a learned rule is:
\begin{lstlisting}
 nationality(A,C) :- bornInCity(A,B), cityOfCountry(B,C).
\end{lstlisting}

The main drawback of \cite{bishan-iclr15} is that they have an {\em ad hoc} rule for conjunction which permits them to only represent conjunctions of ``adjacent'' relations via matrix multiply. They do not have general compositional rules which permit the computation of conjunction of arbitrary formulas. For instance their technique cannot be used to learn clauses such as: 
\begin{lstlisting}
 dualCitizen(A, C, C1) :- 
   bornIn(A,B), cityOfCountry(B, C1), C != C1, 
   residentOfCountry(C), 
   naturalizedInCountry(C).
\end{lstlisting}
With the techniques of this paper we can proceed as follows. The key point is that formulas are polynomials that can be manipulated algebraically. Assume we have identified a first-order formula $\phi(\bar{X})$ as the candidate for the body of a clause with head (an atomic formula) $\psi(\bar{Y})$. Typically $\bar{Y}\subseteq \bar{X}$, consider for the moment that $\bar{X}=\bar{Y}$ (for the more general case we will need an order-reducing tensor operation, corresponding to projection or existential quantification). We need to determine whether the clause $\phi(\bar{X})\rightarrow \psi(\bar{Y})$ is the case. Operating symbolically on the corresponding polynomials, we need to determine whether $1-\phi(1-\psi) \approx 1$, i.e.{} $\phi \approx \phi\psi$. This can be done by tensor operations. 

In essence, we can use the theory presented in this paper to address the problem of rule learning in a more general way.

\subsection{CCP programs with sketches}

\section{Work to do}

How does one deal with infinite domains?

{\em How does one deal with infinite domains? Does one need to? Do the
usual machine learning algorithms make sense over infinite domains?
The representations are all in terms of finite vectors/ matrices /
tensors, and individuals are represented with $n$-vectors, so in any
case we are basically considering only an equivalence class of the
domain with $2^n$ classes.}

{\em Track down whether there is any work in psuodo-Boolean
  optimization related to optimization in the presence of first-order
  constraints.
}

{\em Are there general techniques from polynomial theory to help us do
  logic, e.g. using Grobner basis, as in \cite{Brickenstein:2009:PFG:1550968.1551286}.}

{\em Go back to the work on constrained clustering, adding these kinds
of representations of first-order constraints. But first have to find
a tractable way of dealing with quantifiers. Look to hyper-doctrinal paper as a general 
way to introduce quantifiers.}

{\em Applications in constraint programming?}

\paragraph{Acknowledgements.} Luc de Raedt, Angelika Kimmig, Tom Mitchell, William Cohen, Bishan Yang, Michael Witbrock, Piero Molino, \ldots

\bibliographystyle{alpha}
\bibliography{master}

\appendix
\newpage

\section{Clausal form and logic programming}
\subsubsection{Clausal representation}
The general representation of a clause with literals $a_i,b_j$ is:
\begin{equation}
\LL{a_1,\ldots, a_m \Arrow b_1, \ldots, b_n} = 1-(\Pi_{i \in 1\ldots m} \LL{a_i}) \mtimes (\Pi_{j\in 1\ldots n} \bar{b}_j)
\end{equation}
\noindent (Think: $a_1,\ldots, a_m \Arrow b_1, \ldots, b_n$ is the same as $\neg ((a_1 \wedge \ldots a_m)
\wedge (\neg b_1) \wedge \ldots \wedge (\neg b_n))$.)

Definite clauses, negative clauses, unit clauses, and empty clauses are the special cases with 
$n=1$, $n=0$, $(m=0,n=1)$ and $(m=0,n=0)$: 
  \begin{align*}
    \LL{a_1,\ldots, a_m \Arrow b} &= 1-{(\Pi_{i \in 1\ldots m} a_i) \mtimes \bar{b}}\\
    \LL{a_1,\ldots, a_m \Arrow } &= 1-{\Pi_{i \in 1\ldots m} a_i}\\
    \LL{ \Arrow b} &= b\\
    \LL{ \Arrow } &= 1 - 1 = 0
  \end{align*}

Now we can see how unit resolution works.  Given $a=1$, unit resolution reduces $1-a\mtimes\phi$ to $1-\phi$; given $a=0$, $1- (1-a)\mtimes \phi$ reduces to $1-\phi$. 

More generally, one can think of definite clause programming thus. We are given a theory $\Sigma=\{(1-\phi_1),\ldots, (1-\phi_n)\}$, where each formula $1-\phi_i$ represents a definite clause.  
Let us say that a {\em valuation} is a function from the set of variables to $\{0,1\}$. A {\em solution} of $\Sigma$ assigns $1$ to each formula $1-\phi_i$ in $\Sigma$, i.e.{} assigns zero to each $\phi_i$. 

Now we are given a query $q=a_1 \And \ldots \And a_k$ and asked to establish $\Sigma \vdash q$. That is, every solution for $\Sigma$ must assign $1$ to each $a_j, j\in 1\ldots k$. Backward chaining consists of finding some clause $b_1,\ldots, b_m\Arrow a_j$, for some $j\in 1\ldots k$ and replacing the query $q$ with $a_1\And \ldots \And a_{j-1} \And b_1 \And \ldots b_m \And \ldots \And a_k$. (Note that $m$ could be zero, thus decreasing the size of the resulting query.) This process is repeated until the size of the final query is $0$.

Why is this sound? Because every solution $v$ that assigns $0$ to $b_1\mtimes \ldots \mtimes b_j\mtimes (1-a_j)$ and $1$ to 
$a_1\mtimes \ldots \mtimes a_{j_1}\mtimes b_1 \mtimes \ldots \mtimes b_m \mtimes a_{j+1}\mtimes \ldots \mtimes a_k$ can only do so by 
assigning $1$ to $b_1, \ldots, b_m$ {\em and} $1$ to $a_j$, and $1$ to $a_1,\ldots, a_{j-1},a_{j+1},\ldots a_k$. Therefore it assigns $1$ to $a_1 \mtimes \ldots \mtimes a_k$. 

\paragraph{Posiform.}
%% TODO: Have not introduced pseudo-Boolean yet. Why is this important to say. What does ``many ways'' mean?
Let $\bar{x}$ stand for $(1-x)$. Every pseudo-Boolean function can also be represented in many ways through ``clausal decompositions'', called {\em posiforms}:
\begin{equation}
  f(x_1,\ldots, x_n)=b_0 + \Sigma_{k \in 1\ldots m} b_k (\Pi_{i\in A_k} x_i \Pi_{j\in B_k}(\bar{x_j}))
\end{equation}
\noindent where $b_k\geq 0$ for $k \in 1\ldots m$. 

\section{Brief primer on tensors}\label{sec:tensor}
The material here is based on \cite{lee-book-2000,dullemond-1991-tensor}.


Let \(V\) be an \(n\)-dimensional vector space over the reals. A {\em covector} on \(V\) is a real-valued linear functional \(\omega:V\rightarrow \Re\). The space of all covectors is itself a real vector space  under pointwise addition and multiplication. It is written as \(V^{\star}\) and called the {\em dual space} to \(V\). 

\begin{proposition}[\cite{lee-book-2000} Proposition 4.1] For \(V\) an \(n\)-dimensional vector space and \(E_1, \ldots, E_n\) a basis for \(V\), the covectors \(\epsilon^1, \ldots, \epsilon^n\), defined by:
  \[
  \epsilon^i(E_j) = \delta^i_j = \left\{
  \begin{array}{ll}
    1 & \mbox{if \(i =j\)} \\
    0 & \mbox{if \(i\not=j\)}
    \end{array}\right.
    \]
\noindent form a basis for \(V^{\star}\), called the {\em dual basis} to \((E_i)\).   
\end{proposition}
%% TODO: Define canonical isomorphism.
Note that it follows that the dimensionality of \(V^{\star}\) is the same as that of \(V\). It is also the case that \(V^{\star\star}\) is canonically isomorphic to \(V\). 

Tensors can be represented in one of three equivalent ways -- as multi-dimensional arrays that are invariant under basis change, as multilinear functions of \(V\) and \(V^{\star}\) and as elements of tensor products of \(V\) and \(V^{\star}\). Here we choose the last alternative. 

\begin{definition}[Tensor] For $V$ a finite-dimensional real vector space, a {\em tensor} of rank \((k,l)\) over $V$ is an element of the tensor space \(V\otimes \ldots \otimes V \otimes V^{\star}\otimes \ldots \otimes V^{\star}\) where the first product has \(k\) elements and the second \(l\). \(k\) is said to be its {\em contravariant} rank, and \(l\) its {\em covariant} rank. The {\em rank} of the tensor is \(k+l\).
\end{definition}
Denote the set of all \((k,l)\)-tensors over \(V\) by \(T^k_l\).
$T^k_l(V)$ is a vector space under point-wise addition and scalar multiplication. Specifically:
\[
\begin{array}{l}
  (\alpha T)(X_1, \ldots, X_k, Y_1, \ldots, Y_l) = \alpha(T(X_1, \ldots, X_k, Y_1, \ldots, Y_l))\\
  (S+T)(X_1, \ldots, X_k, Y_1, \ldots, Y_l)=S(X_1, \ldots, X_K, Y_1, \ldots, Y_k) + T(X_1, \ldots, X_k, Y_1, \ldots, Y_l)
\end{array}
\]
Note that a matrix \(M\) is a \((1,1)\)-tensor, a vector \(x\) a \((1,0)\)-tensor, and a covector \{a\} a \((0,1)\)-tensor. \((k,l)\)-tensors are in a 1-1 correspondence with multilinear maps \(V^{\star}\times \ldots \times V^{\star} \times V \times \ldots \times V \rightarrow \Re\), where the first product has \(k\) elements and the second \(l\).

The meaning of the indices of a tensor is usually decided beforehand; it is conventional to write a tensor with its contravariant (upper) indices first and the covariant (lower) indices last, e.g.{} \({T^{ij}}_k\). For instance, a matrix \(M\) is a \((1,1)\)-tensor, and we can write \({M^j}_i\) to make its indices explicit. This is the same as \({M^j}_k\) -- the identity of indices does not matter. Also, basis vectors will be written with a lower index, and components of a vector with respect to this basis with an upper index; similarly, basis covectors will be written with an upper index and components of a covector with a lower index. 


In the following we shall adopt the {\em Einstein summation convention}:
\begin{quotation}
  If the same index name appears twice in any term, once as an upper index and once as a lower index, that term is understood to be summed over all possible values of that index, generally from \(1\) to the dimension of the space in question. If the indices belong to different tensors, their tensor product is taken first.
\end{quotation}
Thus for instance we can rewrite:
\[
\begin{array}{lcl}
  \Sigma_{\nu=1}^n A_{\mu\nu}v^{\nu} & \rightarrow &   A_{\mu\nu}v_{\nu}\\
  \Sigma_{\beta=1}^n \Sigma_{\gamma=1}^n A _{\alpha \beta}B^{\beta \gamma}C_{\gamma \delta} & \rightarrow & A_{\alpha \beta}B^{\beta \gamma}C_{\gamma \delta}
\end{array}
\]

Let \(\{{\bf e}_j\}\) be a basis for \(V\), with canonical cobasis \(\{\epsilon^i\}\) for \(V^{\star}\). 
A \((k,l)\)-tensor \(T\) can be associated with an array with \(k+l\) dimensions, whose \((i_1, \ldots, i_k, j_1, \ldots, j_l)\)'th element, written \(T^{i_1\ldots i_k}_{j_1, \ldots, j_l}\), is given by the coefficient of \(T\) for the basis element \({\bf e}_{i_1}\otimes \ldots \otimes {\bf e}_{i_k}\otimes \epsilon^{j_1}\otimes \ldots \otimes \epsilon^{j_l}\). Thus:
\[ T = T^{i_1 \ldots i_k}_{j_1 \ldots j_l} {\bf e}_{i_1}\otimes \ldots \otimes {\bf e}_{i_k}\otimes \epsilon^{j_1}\otimes \ldots \otimes \epsilon^{j_l}
\]



\subsection{Tensor Contraction}\label{sec:summation-convention}


The {\em contraction} of a tensor is obtained by setting one upper and one lower index equal, thus indicating a summation per the convention above. The result of contracting a \((k,l)\)-tensor  is a \((k-1,l-1)\) tensor.

The contraction operation is invariant under coordinate changes.


\subsection{Working with tensors}
  
Tensors can be ``partially'' evaluated. For \(T\) a tensor of rank \(k\) representing a predicate \(p\), and \(a\) a vector representing the value of argument \(i\), \(T_{1\ldots k}a^i\) represents the predicate \(\lambda x_1, \ldots, x_{i-1},x_{i+1},\ldots x_k. p(x_1, \ldots, x_{i-1}, a, x_{i+1}, \ldots, x_k)\). It can be thought of as {\em contracting} \(T\) and \(a\) on index \(i\).
  

\begin{example}\label{ex:tc-1}
  Consider the tensor representation of the boolean polynomial \(p(x,y) = x(1-y)\). Because \(x,y\in \{0,1\}\) with the two values independent of each other, we will embed them in the two dimensional vector space \(U=\{0,1\}^2\), with ``one hot'' basis vectors \({\bf e}_1=(1,0)\) and \({\bf e}_2=(0,1)\) representing \(0\) and \(1\) respectively. Now the tensor \(T_{i,j}\) representing the binary predicate \(\lambda x,y. p(x,y)\) is represented by the table:
  \[
  \begin{array}{lll}
    {T_{1,1}}&=(0)(1-0)&=0\\
    {T_{1,2}}&=(0)(1-1)&=0\\
    {T_{2,1}}&=(1)(1-0)&=1\\
    {T_{2,2}}&=(1)(0)&=0
  \end{array}
  \]
  We can use this representation to evaluate the predicate at different points through tensor contraction. For instance, \(p(0,1)\) is given, for \(a={\bf e}_1,b={\bf e}_2\) by \(T_{i,j}a^i b^j\), which through the Einstein convention expands out to:\footnote{Recall that for a vector \(x\), \(x^k\) represents its \(k\)th coordinate. Note that
    in the first  term \(T_{i,j}a^ib^j\) the super-scripts of \(a\) and \(b\) are formal (unbound) indices, whereas in  the second term \(T_{i,j}a^i b^j \), \(i\) and \(j\) are bound, and the super-scripts of \(a\) and \(b\) represent component selections.}
  \[\begin{array}{ll}
  T_{i,j}a^i b^j &= \Sigma_{j=1}^2 \Sigma_{i=1}^2 a^i b^j T_{i,j}\\
  &= (1)(0) T_{1,1} + (1)(1) T_{1,2}
      + (0)(0) T_{2,1} + (0)(1) T_{2,2}\\
  & =T_{1,2}\\
  & = 0
  \end{array}
  \]

  Similarly, the term \(T_{i,j} a^i\) gives us a \((1,0)\) tensor \(S_j\) representing \(\lambda x.p(0,x)\):
  \[
  \begin{array}{l}
    S_1 = \Sigma_{i=1}^2 a^i T_{i,1} =a^1 T_{1,1} + a^2  T_{2,1} = T_{1,1} = 0\\
    S_2 = \Sigma_{i=1}^2 a^i T_{i,2} =a^1 T_{1,2} + a^2  T_{2,2} =  T_{1,2} = 0    
  \end{array}\]

  The term \(T_{i,j} a^j\) gives us a \((1,0)\) tensor \(R_i\) representing \(\lambda x. p(x,0)\):
  \[\begin{array}{l}
    R_1 = \Sigma_{i=1}^2 a^i T_{1,i} =a^1 T_{1,1} + a^2  T_{1,2} = T_{1,1} = 0\\
    R_2 = \Sigma_{i=1}^2 a^i T_{2,i} =a^1 T_{2,1} + a^2  T_{2,2} =  T_{2,1} = 1    
  \end{array}\]

    Applying \(R_i\) to \(b\), i.e. computing \(R_ib^i\) gives \(R_1 b^1 + R_2 b^2=(0)(0) + (1)(1)=1\), whereas applying it to \(a\)  gives  \(R_1 a^1 + r_2 b^2 = (0)(1)+(0)(0) = 0\). The results in both case agree with \((\lambda x.p(x,0))(1)\).
  
\end{example}

More generally, we can compute tensor contraction symbolically. Let \(x,y\) be  unknown vectors in \(U\), and let \(S\) be a rank-2 tensor over \(U\). Then if \(S\) represents the predicate \(q\), the predication \(q(x,y)\) is represented by \(S_{i,j}x^i y^j\) which expands out to:
\[\begin{array}{ll}
  [S(x,y)]&\defeq  S_{i,j}x^i y^j\\
  & = \Sigma_{j=1}^2 \Sigma_{i=1}^2 x^i y^j [S_{i,j}] \\
  & =   x^1 y^1  [S_{1,1}] + x^1 y^2 [S_{1,2}]
      + x^2 y^1 [S_{2,1}]  + x^2 y^2 [S_{2,2}]
  \end{array}
\]

\begin{example}[Example~\ref{ex:tc-1} contd]
  Taking the value of \(S\) above to be \(T\), we get:
  \[\begin{array}{ll}    
      [T(x,y)] & = x^2y^1\\
      & = [x=1][y=0]
  \end{array}
  \]
\end{example}

The representation of an arbitrary $k$-ary predicate is similar, a tensor \(T_k\) represented by \(n^k\) numbers, for \(U=\{0,1\}^n\). 

\end{document}
